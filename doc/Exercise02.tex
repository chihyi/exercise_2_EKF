\documentclass[14pt,a4paper]{article}
\usepackage{ucs}
\usepackage{fancyhdr}
\usepackage{listings}
\usepackage{amsfonts}
\usepackage{amsmath}
\usepackage{setspace}
\usepackage{graphicx}
\usepackage{amssymb}
\usepackage{array}
\usepackage{subfigure}
\usepackage{listing}

\pagestyle{fancy}
\renewcommand\thesubsubsection{}
\renewcommand\thesubsection{\alph{subsection})}
\renewcommand\thesection{Exercise \arabic{section}}
\lhead{Exercise 2}
\rhead{Group 4}
\title{\textbf{Sheet 2} \\  \textbf{Motion Models and Robot Odometry}}
\author{Group 4 \\Urs Borrmann, Caner Hazirbas, FangYi Zhi}

\begin{document}
\maketitle
\onehalfspacing

\section{}
	\refstepcounter{subsection}
	\refstepcounter{subsection}
	\subsection{Group Picture}


	\subsection{Graph Visualization}


	\subsection{Specify the odometry model}


\section{}
	\refstepcounter{subsection}
	\refstepcounter{subsection}
	\refstepcounter{subsection}		
	\subsection{Kalman Filter Covariance Ellipse Secreenshot}
		

	\subsection{Kalman Filter with Higher Noise Screenshots}
	\subsection{Noise Prediction for Experimental Setup}	
	
	\subsection{Observation Function and Its Jacobian}
		\subsubsection{Observation Function}
	 	Observed marker pose is calculated with function $h(x)$ (eq. 1). This $h(x)$ observation function predicts the marker pose $z_{pre}$ given $x$, estimated robot world state(eq 2), and $z_{g}$ ,the marker pose in global frame(eq. 3).
		 
		\begin{enumerate}
		\item $\begin{aligned}[t]
		    &&&&&&&&&&&&& z_{pre}=h(x) \\
		    &&&&&&&&&&&&& z_{pre} = (x_{pre}&&y_{pre}&\psi_{pre})^T
		\end{aligned}$
		\item $\begin{aligned}[t]
		    &&&&&&&&&&&&& x = (x_{w}&&y_{w}&\psi_{w})^T
		\end{aligned}$
		\item $\begin{aligned}[t]
		    &&&&&&&&&&&&& z_{g} = (x_{g}&&y_{g}&\psi_{g})^T
		\end{aligned}$
		\end{enumerate}
	
		In order to find the observation, we need to transform the global marker pose to local frame.
		
		If X is homogeneous transformation matrix of $x$, robot pose,
		
		$$	X =	\begin{pmatrix} 
					R & t \\
					0 & 1 
				\end{pmatrix}
			=	\begin{pmatrix}
					\cos\psi_{w}	 &	-\sin\psi_{w} & x_{w}\\	
					\sin\psi_{w} &	\cos\psi_{w}	 & y_{w}\\
					0		 &		0	&	1
				\end{pmatrix}
		$$	
		
		then we can transform any local frame to global frame as follows;
		
				\[\vec{t_g}= X \vec{t_pre} \]
				
		We want to transform from global to local. In order to do that we should take inverse of X transformation matrix;
		$$	X^{-1} =	\begin{pmatrix} 
		
					R^{-1} & -R^{-1}t \\
					0 & 1 
				\end{pmatrix}
			=	\begin{pmatrix}
					\cos\psi_{w}	 &	\sin\psi_{w} & -x_{w}\cos\psi_{w}-y_{w}\sin\psi_{w}\\	
					-\sin\psi_{w} &	\cos\psi_{w}	 &  x_{w}\sin\psi_{w}-y_{w}\cos\psi_{w}\\
					0		 &		0	&	1
				\end{pmatrix}
		$$
		
		
		Now we can compute the local marker position from global marker position;
		
		\[\vec{t_{g}}= \left( \begin{array}{c}
						x_{g} \\ y_{g}\\ 
				\end{array} \right)\] 	
		$$
			\tilde{t_{pre}}  = X^{-1} \tilde{t_{w}}
			= \begin{pmatrix}
					(x_{w} - x_{l})\cos\psi_{w}	 +	(y_{w}-y_{l})\sin\psi_{w}\\ 	
					-(x_{w} - x_{l})\sin\psi_{w} +	(y_{w}-y_{l})\cos\psi_{w}\\
												1
				\end{pmatrix}
		$$
		
		Since yaw angle is always in the global frame, observed yaw angle is
		
		\[ \psi_{pre}=(\psi_{w} - \psi_{g})\]
		
		At the end we get following observation function
		
		$$
		h(x)	= \begin{pmatrix}
					(x_{w} - x_{g})\cos\psi_{w}	 +	(y_{w}-y_{g})\sin\psi_{w}\\ 	
					-(x_{w} - x_{g})\sin\psi_{w} +	(y_{w}-y_{g})\cos\psi_{w}\\
											(\psi_{w} - \psi_{g})
			\end{pmatrix}
		$$
		
		
	\refstepcounter{subsection}
	\subsection{Trajectory}
		
	
	\subsection{Drift on Pose Estimation}
	

\end{document}
