\documentclass[14pt,a4paper]{article}
\usepackage{ucs}
\usepackage{fancyhdr}
\usepackage{listings}
\usepackage{amsfonts}
\usepackage{amsmath}
\usepackage{setspace}
\usepackage{graphicx}
\usepackage{amssymb}
\usepackage{array}
\usepackage{subfigure}
\usepackage{listing}

\pagestyle{fancy}
\renewcommand\thesubsubsection{}
\renewcommand\thesubsection{\alph{subsection})}
\renewcommand\thesection{Exercise \arabic{section}}
\lhead{Exercise 2}
\rhead{Group 4}
\title{\textbf{Sheet 2} \\  \textbf{Motion Models and Robot Odometry}}
\author{Group 4 \\Urs Borrmann, Caner Hazirbas, FangYi Zhi}

\begin{document}
\maketitle
\onehalfspacing

\section{}
	\refstepcounter{subsection}
	\refstepcounter{subsection}
	\subsection{Group Picture}


	\subsection{Graph Visualization}

	Figure \ref{graph:rxgraph} is the screenshot taken on rxgraph, shows the running nodes and topics.\\
	
	\begin{figure}[htbp]
	\centering
	\includegraphics[width=165mm,height=40mm]{rxgraph.png}
  	\caption{Running nodes and published topics}
    \label{graph:rxgraph}
	\end{figure}

\section{}
	\refstepcounter{subsection}
	\refstepcounter{subsection}
	\refstepcounter{subsection}		
	\subsection{Kalman Filter Covariance Ellipse Screenshot}
		Figure \ref{graph:Q_origin} is the screenshot of the covariance ellipse from the original Q matrix visualized by rviz.\footnote{In order to easily compare the difference between the effect of the two diffenrent Q matrices, we took both screenshots at the last moment of the given bagfile.}\\
	
	\begin{figure}[htbp]
	\centering
	\includegraphics[scale=0.5]{Q_origin.png}
  	\caption{Covariance ellipse with the original Q matrix}
    \label{graph:Q_origin}
	\end{figure}
	
	Figure \ref{graph:filtered_pose} shows the screenshot of the estimated two-dimensional trajectory from the given bag file.
	
	\begin{figure}[htbp]
	\centering
	\includegraphics[scale=0.4]{filtered_pose.png}
  	\caption{Estimated two-dimensional trajectory from the given bag file}
    \label{graph:filtered_pose}
	\end{figure}
	

	\subsection{Kalman Filter with Higher Noise Screenshots}
	
		Figure \ref{graph:Q_2Times} shows the screenshot of the covariance ellipse from the modified Q matrix, which drifts two times more in the global x-direction.\\
	
	\begin{figure}[htbp]
	\includegraphics[scale=0.5]{Q_x_2Times.png}
  	\caption{Covariance ellipse with the modified Q matrix}
    \label{graph:Q_2Times}
	\end{figure}
	
		Figure \ref{graph:filtered_pose_modifiedQ} shows the screenshot of the estimated two-dimensional trajectory from the bag file with modified Q matrix.\footnote{There is no difference of the trajectories between the two different Q matrix, since until now, the Q matrix just adjusts the covariance ellipse but not the state vector, hence the trajectory.}
	
	\begin{figure}[htbp]
	\centering
	\includegraphics[scale=0.4]{filtered_pose_modifiedQ.png}
  	\caption{Estimated two-dimensional trajectory from the bag file with modified Q matrix.}
    \label{graph:filtered_pose_modifiedQ}
	\end{figure}
	
	\subsection{Noise Prediction for Experimental Setup}	
	
	One can fly the quadracopter proportional to the marker located on the ground and read the values from sensors. After a while without changing position or orientation of quadracopter, we can observe how much drift we are having. Using observed amount of drift and waited time, one can estimate the noise.
	
	\subsection{Observation Function and Its Jacobian}
	
		\subsubsection{Observation Function}
		
	 	Observed marker pose is calculated with function $h(x)$ (eq. 1). This $h(x)$ observation function predicts the marker pose $z_{pre}$ given $x$, estimated robot world state(eq 2), and $z_{g}$ ,the marker pose in global frame(eq. 3).
		 
		\begin{enumerate}
		\item $\begin{aligned}[t]
		    &&&&&&&&&&&&& z_{pre}=h(x) \\
		    &&&&&&&&&&&&& z_{pre} = (x_{pre}&&y_{pre}&\psi_{pre})^T
		\end{aligned}$
		\item $\begin{aligned}[t]
		    &&&&&&&&&&&&& x = (x_{w}&&y_{w}&\psi_{w})^T
		\end{aligned}$
		\item $\begin{aligned}[t]
		    &&&&&&&&&&&&& z_{g} = (x_{g}&&y_{g}&\psi_{g})^T
		\end{aligned}$
		\end{enumerate}
	
		In order to find the observation, we need to transform the global marker pose to local frame.
		
		If X is homogeneous transformation matrix of $x$, robot pose,
		
		$$	X =	\begin{pmatrix} 
					R & t \\
					0 & 1 
				\end{pmatrix}
			=	\begin{pmatrix}
					\cos\psi_{w}	 &	-\sin\psi_{w} & x_{w}\\	
					\sin\psi_{w} &	\cos\psi_{w}	 & y_{w}\\
					0		 &		0	&	1
				\end{pmatrix}
		$$	
		
		then we can transform any local frame to global frame as follows;
		
				\[\vec{t_g}= X \vec{t_{pre}} \]
				
		We want to transform from global to local. In order to do that we should take inverse of X transformation matrix;
		$$	X^{-1} =	\begin{pmatrix} 
		
					R^{-1} & -R^{-1}t \\
					0 & 1 
				\end{pmatrix}
			=	\begin{pmatrix}
					\cos\psi_{w}	 &	\sin\psi_{w} & -x_{w}\cos\psi_{w}-y_{w}\sin\psi_{w}\\	
					-\sin\psi_{w} &	\cos\psi_{w}	 &  x_{w}\sin\psi_{w}-y_{w}\cos\psi_{w}\\
					0		 &		0	&	1
				\end{pmatrix}
		$$
		
		
		Now we can compute the local marker position from global marker position;
		
		\[\vec{t_{g}}= \left( \begin{array}{c}
						x_{g} \\ y_{g}\\ 
				\end{array} \right)\] 	
		$$
			\tilde{t_{pre}}  = X^{-1} \tilde{t_{g}}
			= \begin{pmatrix}
					(x_{g} - x_{w})\cos\psi_{w}	 +	(y_{g}-y_{w})\sin\psi_{w}\\ 	
					-(x_{g} - x_{w})\sin\psi_{w} +	(y_{g}-y_{w})\cos\psi_{w}\\
												1
				\end{pmatrix}
		$$
		
		Since yaw angle is always in the global frame, observed yaw angle is
		
		\[ \psi_{pre}=(\psi_{w} - \psi_{g})\]
		
		At the end we get following observation function
		
		$$
		h(x)	= \begin{pmatrix}
					(x_{g} - x_{w})\cos\psi_{w}	 +	(y_{g}-y_{w})\sin\psi_{w}\\ 	
					-(x_{g} - x_{w})\sin\psi_{w} +	(y_{g}-y_{w})\cos\psi_{w}\\
											(\psi_{w} - \psi_{g})
			\end{pmatrix}
		$$
		
		\subsubsection{Jacobian of Observation Function}
		
			Now we can  compute the jacobian of observation function as following;
		
		$$
			H = \frac{\partial h(x)}{\partial x}
			  = \begin{pmatrix}
			  \frac{\partial h(x)}{\partial x_{w}} &
			  \frac{\partial h(x)}{\partial y_{w}} &
			  \frac{\partial h(x)}{\partial \psi_{w}}  			
			  \end{pmatrix}
		$$
		
		$$
			  = 
			  \begin{pmatrix}
					-\cos\psi_{w} & -\sin\psi_{w} & -(x_{g} - x_{w})\sin\psi_{w} +	(y_{g}-y_{w})\cos\psi_{w}\\  	
					\sin\psi_{w}  &	-\cos\psi_{w} & -(x_{g	} - x_{w})\cos\psi_{w} -	(y_{g}-y_{w})\sin\psi_{w}\\
											0 & 0 & -1
			\end{pmatrix}
		$$
	\refstepcounter{subsection}
	\subsection{Trajectory}
		Figure \ref{graph:filtered_pose_corrected} shows the screenshot of the EKF corrected two-dimensional trajectory from the given bag file.
	
	\begin{figure}[htbp]
	\centering
	\includegraphics[scale=0.4]{filtered_pose_corrected.png}
  	\caption{From the EKF corrected two-dimensional trajectory from the given bag file}
    \label{graph:filtered_pose_corrected}
	\end{figure}
		
	
	\subsection{Drift on Pose Estimation}
	
	At the end of the 48th second, roughly we have 1.5m drift in x direction and 1m in y direction.

\end{document}
